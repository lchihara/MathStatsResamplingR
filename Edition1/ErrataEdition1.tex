\documentclass[11pt]{article}

\setlength{\parindent}{0in}
\input{ChapPrelim2}
%%%%%%%%%%%%%%%%%%%%%

%For postscript
%\usepackage[pdftex]{graphicx}

\usepackage[pdftex]{graphicx} \DeclareGraphicsExtensions{.pdf,.png,.jpg,.mps}
\graphicspath{{.}{./Images/}{../Images/}}
\usepackage{fancyhdr}
\usepackage{color}
\pagestyle{fancy}
\usepackage{comment}

\lhead{{\em Mathematical Statistics with Resampling and \R}, (First Edition).
By Laura Chihara and Tim Hesterberg}

\begin{document}

\medskip
{\bf \Large Errata} (11 January 2018)

\bigskip
%--------------------------------------------------
% Preface
% Corrected 6 October 2011
Page xiii, paragraph 2, change Waldrop to \textcolor{red}{Wardrop}.

\bigskip
%--------------------------------------------------
%Added 27 April 2012
Page 6, First sentence in Section 1.7. {\color{red}{Whom}} do you plan to vote for...


\bigskip
%Added 6/2/2016
Page 15, Table 2.3: missing an interval $\textcolor{red}{(300, 350]}$, with frequency
\textcolor{red}{2}.


\bigskip
%Added 7 January 2012
Page 23, line 1 and 3.% The pdf should be $f(x)= \textcolor{red}{3}e^{-3x}$:

\smallskip
Alternatively, since we know the pdf of $X$ is $f(x)=\textcolor{red}{3}e^{-3x}, x\ge 0$, we
could also solve for $q_p$ in

\[
p = P(X \le q_p)=\int_0^{q_p}\textcolor{red}{3}e^{-3t}dt.
\]


\bigskip
%Added 7 January 2012
Page 26, {\bf \R~Note}: line 11. \\
Delete: \cmd{abline(v = 25, col= "red")}.\\

Page 26: 4 lines above bottom of box, the legend command should be \\
\cmd{legend(\textcolor{red}{"topleft"}, legend=c("Males", "Females"), ...)}

\bigskip
%Added 14 Dec 2012
Page 29, 2nd line above images: \\
while a variable with negative \textcolor{red}{kurtosis} is flatter...

\bigskip
%Added 8 Jan 2013
Page 30, Exercise 3(c):\\
 Are there any conditions that would ensure that
$f(\xbar)$ is the \textcolor{red}{mean} of the transformed data?

\bigskip
%Added 7 January 2012
Page 42, Figure 3.3 is incorrect
\begin{center}
\includegraphics[scale=.2, angle=-90]{ch03fg3-corrected}
\end{center}


\bigskip
%Added 14 December 2013
Page 43 {\bf Remark:} First line\\
One subtle point is that the transformation need\textcolor{red}{s} to be...

Page 43 Third line of {\bf Remark}: \\
 example, we used $p=P(\Xbar_1-\Xbar_2 \ge \textcolor{red}{\xbar_1-\xbar_2})$.

\bigskip
%Added 6/2/2016
Page 52, second equation $P(T \ge 5) \cdots \ = P(\textcolor{red}{Z} \ge 4.975)$

%Added 7 January 2012
\bigskip
Page 53, line 4 from bottom: \\
For instance, the expected count for the (4, 2)-cell is
$87\times 409/1307= \textcolor{red}{27.2249}$.\\

Page 53, Table 3.4 The row corresponding to Graduate should be:\\
\begin{tabular}{rrrrr}
Graduate & 64 & 50 & \textcolor{red}{114} & 56.6\%
\end{tabular}

\bigskip
%Added 7 January 2012
Page 54, Table 3.5

\begin{tabular}{rrr}
...&&\\
HS & 488.5065 & \textcolor{red}{222.4935}\\
JrColl& 59.7751 & \textcolor{red}{27.2249}
\end{tabular}

\bigskip
%Added 7 January 2012
Page 58 {\bf\R~Note}, line 4: \\
$>$ \cmd{1 - pchisq(23.45, \textcolor{red}{4})}

\bigskip %14 March 2013
Page 65, Table, 2nd line:

\begin{tabular}{rrrrr}
Observed count & 30 & 30 &  \textcolor{red}{22}& 18\\
\end{tabular}




\bigskip
Page 65, last unnumbered equation:
\[
c = \frac{(30-22)^2}{22}+\frac{(30-30.6)^2}{30.6} +
\frac{(22-\textcolor{red}{18.6})^2}{\textcolor{red}{18.6}}+\frac{(18-28.7)^2}{28.7}
=\textcolor{red}{7.53}
\]
Under the null hypothesis, the test statistic comes from a chi-square
distribution with
4-1 degrees of freedom, so the $P$-value is $P(c\ge 7.53)= \textcolor{red}{0.056}$,
\textcolor{red}{so it is plausible that
the data do come from $\Exp{1}$}.

\bigskip %19 January 2014
Page 67, line 2. $f(x) = P(X = x)=\lambda^x e^{-\textcolor{red}{\lambda}}/x!, x =0, 1, 2,...$


\bigskip %April 2014
Page 68, line 4: \\
The chi-square statistic is 0.84; the $P$-value is $P(\chi_3^2 > \textcolor{red}{0.84}) = 0.84$.

\bigskip %11 January 2018
Page 86, Example 4.8\\
$z=3.917$ so $P(Z \ge 3.917) = 0.00004$.

\bigskip %30 March 2017
Page 90, Example 4.12\\
Solution: Then $X \sim \textcolor{red}{\Binom{700}{0.229}}$ with expected value...

\bigskip %15 March 2013
Page 91 starting with Equation (4.5):
\[
P\bigl(\frac{\Xbar-\mu}{\sigma/\sqrt{n}}\le z\bigr)
\approx \Phi(z) \textcolor{red}{-} \frac{\kappa_3}{6\sqrt{n}}(z^2-1)\Phi^\prime(z)
\]
%Added 27 April 2012
where $\Phi(z)$ is the standard normal {\color{red}{CDF}}, $\Phi^\prime(z)$ is its
derivative, and $\kappa_3=E\textcolor{red}{[}(X-\mu)^3\textcolor{red}{]}/\sigma^3$.


\bigskip
%Added 7 January 2012
Page 97, Exercise 29(c). Delete $\Expect{(X-\mu)^3}=1/160$, add period after the word
density.

\bigskip
%Added 7 January
Page 130, Exercise 6(a)\\
 with $k_1$ occurrences of $a_1$,
$k_2$ occurrences of \textcolor{red}{$a_2$},....

\bigskip
%Added 12 January 2015
Page 131, line 1: \\
Use simulation \textcolor{red}{(with n = 200)} to generate...

\bigskip
%Added 12 January 2015
Page 132, exercise 14(a)\\
baby girls born in Wyoming and \textcolor{red}{Alaska}...\\
exercise 14(d) Conduct a permutation test to \textcolor{red}{see if the
difference in mean weights is statistically significant.}

\bigskip %April 2014
Page 136, 5 lines above Definition 6.1: \\
..., compute the derivative
$\ds L^\prime(p) = 5p^4(1-p)^3 + p^5 3\textcolor{red}{(1-p)^2}(-1)$ and set this...


\bigskip
%Added 8 Sept 2011
Page 139, line 2 from bottom: \\
from the exponential distribution with pdf $f(x; \lambda)=\lambda e^{\textcolor{red}{-}\lambda x}$.

\bigskip %Added 12 March 2014
Page 143, Equation (6.8)\\
\[
\frac{\partial(\ln(L(k,\lambda)))}{\partial \sigma}=
-\frac{n}{\sigma} + \frac{1}{\textcolor{red}{\sigma^3}}\sum_{i=1}^n(x_i-\mu)^2=0.
\]


\bigskip
Page 145, Equation (6.10)

\[
  \frac{\partial(\ln(L(k,\lambda)))}{\partial k}
  = \frac{n}{k} - n\ln({\color{red}{\lambda}}) +\sum_{i=1}^n \ln(x_i)-\sum_{i=1}^n
     \bigl(\frac{x_i}{\lambda}\bigr)^k\ln(\frac{x_i}{\lambda})=0
\]


\bigskip
Page 145, Equation (6.13)% the middle term needs a factor $(1/n)$:
\[
\frac{1}{k}+{\color{red}{\frac{1}{n}}}\sum_{i=1}^n \ln(x_i) - \frac{1}{\alpha}\sum_{i=1}^n x_i^k\ln(x_i)=0
\]


Page 152, Example 6.13 second paragraph, second line:\\
 samples of size 25 from $\Unif{0}{12}$. For each sample, we will compute
 $2\xbar$ and $\textcolor{red}{26/25}\times $...\\

Page 152 {\bf \R~Note}:, 7 lines down
\begin{verbatim}
my.max[i] <- 26/25 * max(x)
\end{verbatim}


%--------------------------------------------------
% \bigskip {\bf Chapter 7: Classical Inference: Confidence Intervals}
\bigskip
Page 170 Example 7.2, item 1:\\
... cars in this company is between \textcolor{red}{29.5} and 33.4 mpg.

%Added 25 March 2015
Page 178 Paragraph starting with Let $X$ and $Y$...\\
$\ds X-Y \sim \Norm{\mu_1-\mu_2}{\sigma_{\textcolor{red}{1}}^2 + \sigma_2^2}$

Page 178 Equation (7.9)\\
$\ds \Norm{\mu_1-\mu_2}{\frac{\sigma_1^2}{n_1}}+\frac{\sigma_{\textcolor{red}{2}}^2}{n_2}$.

\bigskip
Page 190 {\bf R Note}, last line:\\
(the interval is $[\textcolor{red}{3422.98}, \infty)$).


\bigskip%Added 2 April 2013
Page 191 Non-numbered equation above (7.17):
\[
\bigl(\frac{n}{1.96^2}+1\bigr)p^2 - \bigl(\frac{2n\phat}{1.96^{\textcolor{red}{2}}}
\textcolor{red}{+ 1}\bigr)p + \frac{n\phat^2}{1.96^{\textcolor{red}{2}}}=0
\]


\bigskip
Page 193,{\bf Remark (2nd bullet)}: \\
The center of the score interval is \textcolor{red}{$(\phat + q^2/(2n))(1+q^2/n)$}. If we set...

\bigskip
Page 194, {\bf Solution}, 2nd equation:\\ % $0.05$ under radical sign should be $0.5$:
radical sign (see Exercise 22).
\[
1.96\sqrt{\frac{{\color{red}{0.5(1-0.5)}}}{\tilde{n}}}\le 0.04
\]


\bigskip %14 March 2013
Page 196, line 4:\\
assume that the statistic $T= (\textcolor{red}{\Xbar}-\mu)/(S/\sqrt{n})$ follows...

\bigskip %Added 14 March 2013
Page 208 \# 34. see Exercise 11, second line:\\
4 degrees of freedom (see Exercise 11 \textcolor{red}{in Appendix B}). Use...

\bigskip %Added 12 January 2014
Page 226 {\bf Solution}, last equation:\\
\[
P(Y\ge 5 \,|\, \theta=2) = \sum_{k=5}^{8}\binom{8}{k}(0.3185)^k(1-0.3185)^{8-k}
= \textcolor{red}{0.0736}
\]

\begin{verbatim}
> sum(dbinom(5:8, 8, 0.3185))
> 1- pbinom(4, 8, 0.3185)     #alternatively...
\end{verbatim}

\bigskip %Added 23 Feb 2014
Page 244 Exercises \# 28, 29, last sentence in both: \\
Include a graph similar to Figure \textcolor{red}{8.5}.

%--------------------------------------------------
% \bigskip {\bf Chapter 9: Regression}

\bigskip
Page 270, Theorem 9.4 item 4. \\
$\ds \Var{\hat{\alpha}}=
\sigma^2\bigl[1/n + {\color{red}{\xbar^2}}/ss_x\bigr]$

\bigskip
Page 276 {\bf Solution}, first equation:
\[
\ds 11.36\sqrt{1 + \frac{1}{24} + \frac{\textcolor{red}{199.7134}}{2195.61}} \approx 12.09
\]

\bigskip
Page 294, Exercise 5, second line:\\
$\Var{Z}=3$ and $\Cov{X}{Y}=-2$, $\Cov{X}{\color{red}{Z}}= -4$, and $\Cov{Y}{Z}=7$.

\bigskip
Page 295, Exercise 14, 3rd line\\
is more \textcolor{red}{than} 5\%.

%---------------------------------------
%\bigskip {\bf Chapter 10: Bayesian Methods}
%Added 2 Feb 2017
\bigskip
Page 304, First displayed equation

\[
 = \frac{(0.03)\color{red}{(0.128)}}{P(\mbox{WLL})}
\]
\bigskip
Page 309, Top Box:\\
The {\color{red}{posterior}} distribution is proportional...
.

\bigskip %April 2014
Page 318, 2 lines above {\bf Remark}:\\
 quantiles of $N(45.53, \textcolor{red}{1.953^2})$: the probability that the true...


\bigskip
Page 325, Exercise 9, line 3: \\
$\mu \sim N(0.72, \textcolor{red}{0.08^2})$. He measures the BMD in ...


%--------------------------------------------------
% \bigskip {\bf Chapter 11: Advanced Topics}

\bigskip
Page 337, line 1:\\
case is $\Ybar/\Xbar$ as an estimator for  $r = \mu_{\textcolor{red}{Y}} / \mu_{\textcolor{red}{X}}$.

\bigskip
Page 337, the paragraph before Equation (11.6) should read:

Using only the first approximation in Equation 11.5,
% $ \Ybar/\Xbar \approx \mu_Y$,   % (This line in the first edition)
$ \Ybar/\Xbar \approx \textcolor{red}{r = \mu_Y/\mu_X}$, % This line in the errata
suggests that the estimate is consistent.

\bigskip
%Added 20 January 2015
Page 344 3rd line from bottom:\\
and then let $h(x) = \sin(x^2)\exp{(-x^2+\textcolor{red}{x})}$.

\bigskip
Page 344, last line:\\
domain, ..., and $f\textcolor{red}{(x)}=\exp{-x}I(x > 0)$

%\bigskip
%Page 337, after Equation (11.6), the next sentence:\\
%\textcolor{red}{For paired data,}  the estimate is asymptotically
%normal with mean r and...

\bigskip
%Added 29 April 2012
Page 351 Equation (11.20)
\[
g(x)=\begin{cases} \lambda \exp{(-\lambda{\color{red}{(x-700)}})} & x \ge 700\\
        0 & x < 700
     \end{cases}
\]

\bigskip
%Added 11 January 2018
Page 356, 1st paragraph, 3rd line:\\
This value is $-200^{\color{red}2}(1/110 + 1/90)$.


%--------------------------------------------------
%Appendix A: Probability Review

\bigskip
Page 364, Example A.1\\
\[
F(x)=P(X\le x) =
\int_0^x {\color{red}{\lambda}} e^{-\lambda t}\, dt = 1-e^{-\lambda x}
\]

%--------------------------------------------------
% \bigskip {\bf Appendix B: Probability Distributions}

\bigskip  %Added 14 March 2013
Page 394, Exercise 14\\
Prove that the expected value of $X \sim F_{m, n}$ is ${\color{red}{n}}/(n-2)$ for $n > 2$.

%Appendix
\bigskip %Added 2 April 2013
Page 396, Under {\bf Binomial}:\\
$f(x; n, p) = \binom{n}{x}p^x (1-p)^{\textcolor{red}{n}-x}$\\

Page 396, Under {\bf Geometric}:\\
$f(x;  p)= (1-p)^{\textcolor{red}{x-1}}p$

\bigskip
Page 398, Under {\bf Gamma}, last column:\\
$(1-t/\textcolor{red}{\lambda})^{-r}$\\

Page 398, Under {\bf Uniform}, last column\\
$\frac{e^{bt}-e^{at}}{(b-a)\color{red}{t}}$


%--------------------------------------------------

\newpage
{\bf Solutions to Odd Exercises}

\medskip

Page 399, Chapter 2 \# 3 (d):
$f$ is an increasing (or decreasing) function and $n$ is odd,
\textcolor{red}{or $f$ is linear}.

\bigskip
%Added 7 January 2012
Page 399, Chapter 2 \#5(a) Favor: \textcolor{red}{899}, Oppose \textcolor{red}{409}


\bigskip
%Added 7 January 2012
Page 400 Chapter 2, \# 15.  Delete entire line (so last solution given for
Chapter 2  is for \# 13.)


\bigskip
Page 400, Chapter 3
%Added 7 January 2012
1.(b) The $P$-value is $\textcolor{red}{2}/10 = 0.2$.

\bigskip
Page 401, 21 $c=8.5819$, $P$-value $= 0.724$.

Page 401, 23(a) Last sentence: ``Conclude that the data do not come from $N(\textcolor{red}{25}, 10^2)$."

\bigskip
Page 401, Chapter 4

%Added 9 January 2012
Page 401 3(a): Sampling distribution of $X+Y$ is
$\{6, 8, 8, 9, 10, 10, 10, 11, 12,\textcolor{red}{12, 13}, 14\}$.\\
11. \textcolor{red}{$n=90$}.\\

\bigskip
Page 401 Chapter 4: \\
{\bf
The numbering is off---delete the current \# 17 (that is, delete 17. (c) 0.506)
and renumber those following by 17, 19, 21, 23, 25, 27.}

\bigskip
%Added 12 March 2014
Page 402, Chapter 5 \# 17(c) \\
1\textcolor{red}{.}63, SE = 0.319. (d) (1.17, 2.22).

\bigskip
Page 402, Chapter 6 \# 17. \\
Shape = 0.917, scale = 17.344,
C = \textcolor{red}{14.217, so times between successive earthquakes do
not follow the Weibull distribution}.

\bigskip
Page 402, Chapter 6 \# 27(b) \\
$(\sigma^4/n^2)\color{red}{2(n-1)}$.


\bigskip %Added 11 February 2014
Page 403, Chapter 6 \# 33b Bias: $-17/(27\theta)$, MSE: $589/(2\cdot 9^3\theta^2)$

\bigskip
Page 403, Chapter 7 \# 7 \\
\textcolor{red}{118.01}

\bigskip
Page 403, Chapter 7, \# 9\\
\textcolor{red}{(28.34, 33.53) cm}.

\bigskip %Added 17 Feb 2013
Page 403, Chapter 7 \# 13(b) $(\textcolor{red}{11.46664}, \infty)$.
We are 95\% confident that, on average,
seedlings grown in fertilized plots grew at least 11.5cm more than seedlings grown
in non-fertilized plots.

\bigskip %Added 19 Feb
Page 403, Chapter 7 \# 21(a)\\
\textcolor{red}{1064}

\bigskip %Added 14 March 2013
Page 404, Chapter 8 \# 23(b)\\
\textcolor{red}{0.473}

\bigskip
Page 404, Chapter 9 \# 3: \\
\textcolor{red}{133}

\bigskip
Page 405, Chapter 10 \# 15 (a)\\
 $f(\theta)=1/\theta^n$, \textcolor{red}{where $\theta > \mbox{max}\{X_1, X_2, \dots, X_N\}$}.
(b)  Pareto distribution with parameters \textcolor{red}{$\alpha+N$, where
$\theta >\mbox{max} \{\beta, X_1, X_2, \dots, X_N\}$}. (c) $0.17$.

\bigskip
Page 405,  Chapter 10 \# 17. (a) The typesetting for the exponential is bad. Should be closer to
$\theta^n\, e^{\theta\sum_{i=1}^n X_i}$. Also, the answer labeled (c) should be labeled (d).

\begin{verbatim}
\theta^n\, e^{\theta\sum_{i=1}^n X_i}
\end{verbatim}
%--------------------------------------------------

\bigskip
{\bf Bibliography}
% References

Page 411, last line, change Waldrop to \textcolor{red}{Wardrop.}

\bigskip
\hrule
\bigskip

\paragraph{Acknowledgements}
Thanks to Sandy Weisberg, Robert Hayden, Charles VanBuskirk, Bob Dobrow,
Kristin Jehring, Elaine Newman, Katie St.~Clair, John Hohwald, Paul Bamberg, Anne Grosse,
Susan Coombs, Risako Owan, Eric Nordmoe, Kirk Goff, Ben Levy, Sushmit Roy, Jaehyo Rhee, Bruce McCullough
and Kirk Goff for providing corrections.

\end{document}


% The second approximation,
% \begin{equation}                        \label{eq:ratioLinear}
%         \Ybar/\Xbar
%         \approx r (1 + \Delta_Y - \Delta_X)
%         = r + (\Ybar - r \Xbar) / \mu_X
% \end{equation}
\noindent
is a constant plus the difference of two sample means
(times constants).  The mean of the expression is $r$,
and variance is
$\Var{\Ybar - r \Xbar}/(\mu_X^2)$.
We consider two important special cases.
In the case of two independent samples, the variance
is $(\sigma^2_Y/n_Y + r^2 \sigma^2_X/nX)/(\mu_X^2)$.
In the case of paired bivariate observations, the
variance is $\Var{Y - r X}/(n \mu_X^2)$.
